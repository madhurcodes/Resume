% % (c) 2002 Matthew Boedicker <mboedick@mboedick.org> (original author) http://mboedick.org
% % (c) 2003-2007 David J. Grant <davidgrant-at-gmail.com> http://www.davidgrant.ca

% %This work is licensed under the Creative Commons Attribution-Noncommercial-Share Alike 2.5 License. To view a copy of this license, visit http://creativecommons.org/licenses/by-nc-sa/2.5/ or send a letter to Creative Commons, 543 Howard Street, 5th Floor, San Francisco, California, 94105, USA.





\documentclass{article}

\usepackage{verbatim}
\usepackage{hyperref}
\hypersetup{
  colorlinks=true,
  urlcolor=blue,      % color of file links
  }
%\usepackage[scaled]{helvet}
%\renewcommand*\familydefault{\sfdefault}

\newenvironment{longversion}{}{} % use this to show longversion
%\newenvironment{longversion}{\comment}{\endcomment} % use this to hide  longversion
\newenvironment{shortversion}{}{} % use this to show shortversion
%\newenvironment{shortversion}{\comment}{\endcomment} % use this to hide shortversion

\usepackage{changepage}
\usepackage{tabularx}
\usepackage{setspace}
\usepackage{url}
\usepackage{sectsty}
\usepackage[letterpaper,margin=0.55in]{geometry}
\pagestyle{empty}

% indentsection style, used for sections that aren't already in lists
% that need indentation to the level of all text in the document
\newenvironment{indentsection}[1]%
{\begin{list}{}%
    {\setlength{\leftmargin}{#1}}%
    \item[]%
}
{\end{list}}

% opposite of above; bump a section back toward the left margin
\newenvironment{unindentsection}[1]%
{\begin{list}{}%
    {\setlength{\leftmargin}{-0.5#1}}%
    \item[]%
}
{\end{list}}

% format two pieces of text, one left aligned and one right aligned
\newcommand{\headerrow}[2]
{\begin{tabular*}{\linewidth}{l@{\extracolsep{\fill}}r}
    #1 &
    #2 \\
\end{tabular*}}

% make "C++" look pretty when used in text by touching up the plus signs
\newcommand{\CPP}
{C\nolinebreak[4]\hspace{-.05em}\raisebox{.22ex}{\footnotesize\bf ++}}

%edit the section font and style
\sectionfont{\normalfont\sectionrule{0pt}{0pt}{-4pt}{1pt}}

%make all sections cap and first letter capital
\newcommand{\tmpsection}[1]{}
\let\tmpsection=\section
\renewcommand{\section}[1]{\tmpsection*{\textsc{#1}}}

%set the line spacing
\setstretch{1.10}


\begin{document}
%TITLE
%todo change the email address and website address to your new address.

\begin{center}
 {\Large \textsc{Madhur Singhal} }\\ 
\begin{tabular}{ l p{4cm} r }
    & &   \\
  Computer Science and Engineering & & msinghal1998@gmail.com \\
  Indian Institute of Technology, Delhi & & cs1150235@cse.iitd.ac.in \\
  & & github.com/madhurcodes \\ & & https://www.linkedin.com/in/madhur-singhal/\\
\end{tabular}
\end{center}


\section{Academic Details}

\begin{center}
\begin{tabular}{ |c | c | c | c |}
\hline
Year & Degree & Institute & CGPA/Percentage \\ 
\hline
2015-2019 & B.Tech in Computer Science & Indian Institute of Technology & 8.69 \\ 
(Current) & and Engineering & Delhi & \\
\hline

2015 & Class XII, CBSE & Pragati Vidya Peeth & 93.8\% \\ 

\hline
2013 & Class X, CBSE & Little Angels High School & 10.00 \\  \hline
\end{tabular}
\end{center}


\section{Scholastic Achievements}
\begin{itemize}
    \setlength\itemsep{0.0em}
    \item Secured \textbf{All India Rank 85} in Joint Entrance Exam Advanced - 2015 among 150 Thousand candidates.
    \item One of three students from India invited by Microsoft to the \textbf{Build 2017} summit in Seattle, Washington.
    \item Runner up in \textbf{Microsoft's Code.Fun.Do} campus wide Hackathon in 2017.   
    \item \textbf{SAT} score 2280/2400 (M:800, R:800, W:680), SAT Subject Tests score 1600/1600 (P:800, M2:800). 
    \item Became a National Talent Search Examination (\textbf{NTSE}) scholar for being in top 1000 at National level in 2011.
\end{itemize}

\section{Major Projects}
% \begin{spacing}{1}
\begin{list} {\labelitemi}{\leftmargin=0em}
\setlength{\leftmargin}{0pt}

    \item[]
    \headerrow {\textbf{Automated Video Description using Deep Learning}}{Prof. Subhashis Banerjee, May-July 2017}
    \begin{itemize}
    \setlength\itemsep{0.0em}
        \item Built software for generating natural language descriptions of short video clips.
		\item Designed encoder decoder network architecture consisting of Multilayered LSTMs to achieve this translation        
        \item Used transfer learning in encoder by employing state of art CNN (Inception V4) trained on Imagenet to encode individual video frames.        
        \item Experimented with Data Augmentation, Audio Features, Attention models and  Alternate Loss metrics to improve performance.
        \item Explored its applications in areas like Video Surveillance and helping Visually impaired.
    \end{itemize}
    \item[]
    \headerrow {\textbf{Indoor Navigation System for Visually Impaired}}{Prof. M. Balakrishnan, May-July 2016}
    \begin{itemize}
    \setlength\itemsep{0.0em}
        \item Designed to help visually impaired people navigate inside buildings and airports.
        \item Uses fingerprinting of Wi-Fi and Bluetooth signals to achieve localization within 2 meters.
        \item Built two Android Applications, one for logging Wi-Fi and BLE strengths and the other to help the user navigate.
        \item Implemented Levenshtein Distance Algorithm, WKNN Algorithm and also tried using  PCA and Isomap in order to reduce dimensionality.
    \end{itemize}


    \item[]
    \headerrow {\textbf{Pipelined MIPS Simulator with Debugger and Cache simulator}}{Prof. Kolin Paul, Mar-April 2017}
    \begin{itemize}
    \setlength\itemsep{0.0em}
        \item Developed a pipelined MIPS simulator supporting animation of instruction execution through multiple stages in C.
        \item Simulated all stages in parallel using threads (pthreads).
        \item Designed a trace based cache simulator and debugger for the processor with various configuration options.
        \item Used SVG to show current instruction in each stage and Javascript, CSS for styling.
    \end{itemize}

\end{list}
% \end{spacing}


\begin{longversion}
\section{Other Projects}
\begin{list} {\labelitemi}{\leftmargin=0em}
\setlength{\leftmargin}{0pt}

% \setlength\itemsep{5em}
%\begin{itemize}

% \item[]
% \headerrow {\textbf{Multicycle ARM Processor}}{Prof. Anshul Kumar, Mar-April 2017}
% \begin{itemize}
% \setlength\itemsep{0.0em}
%     \item Designed Multicycle ARM processor in VHDL that ran on FPGA board.
%     \item Used AHB Lite bus to connect Memory and Input Output interfaces.
%     \item Implemented 7 segment interface for processor to display cycle counts and features like interrupt, reset.
% \end{itemize}
\item[]
\headerrow {\textbf{Study of Data Science and Application to Kaggle Datasets}}{Independent Study, Jan-August 2017}
\begin{itemize}
\setlength\itemsep{0.0em}
    \item Pursuing the Microsoft Professional Program in Data Science having completed 8 of the 10 courses.
    \item Learnt Data Visualization in PowerBI, Data Cleaning and Manipulation in R and Python, and Machine learning with Scikit-Learn and AzureML.
   % \item Completed Andrew Ng`s ML course and read the Elements of Statistics book.
    \item Studied Deep Learning architectures and implemented fully connected and convolutional layers from scratch in Python. 
    \item Completed Kaggle machine learning challanges employing data cleaning, feature engineering and predictive ML algorithms on provided Datasets.
\end{itemize}


\item[]
\headerrow {\textbf{Automated Image Captioning}}{Prof. Subhashis Banerjee, Jan-April 2017}
\begin{itemize}
\setlength\itemsep{0.0em}
    \item Developed a software to automatically generate captions for images.
    \item Used a encoder decoder network similar to machine translation for generating captions.
    \item Used Inception V4 network to extract features from images using transfer learning
    \item Used Multilayered LSTM network to decode image embeddings into natural language sentence.
    \item Achieved baseline performance of paper Show and Tell by Vinyals et al.
\end{itemize}


\item[]
\headerrow {\textbf{Prolog Interpreter in Ocaml}}{Prof. Sanjiva Prasad, Mar-April 2017}
\begin{itemize}
\setlength\itemsep{0.0em}
    \item Developed a Prolog Interpreter in OCaml with full command line interpreter.
    \item Token generation and Parsing was done using OCaml-Lex and OCaml-Yacc respectively.
    \item Rule unification and backtracking were done in order to implement the relational backbone of the interpreter.
\end{itemize}

\end{list}

%\end{itemize}

% \pagebreak

\end{longversion}

\begin{longversion}
\section{Relevant Courses}
\begin{itemize}
\setlength\itemsep{-1em}
\item \textbf{Computer Science:} \\ 
Computer Vision*, Algorithm Design*, Networks*, Programming Languages, Computer Architecture, Design Practices, Data Structures \& Algorithms, Discrete Mathematics, Digital Logic\\

\item \textbf{Mathematics and Electrical Engineering:} \\
Signals \& Systems*, Probability \& Stochastic Processes, Calculus, Linear Algebra, Intro to Electrical Engineering.\\

\item \textbf{Online:} \\
Deep Learning (Fast.ai), Data Science (10 courses on Edx) Intro to Machine Learning (Stanford, Coursera), Intro to Computer Science (CS50, Harvard).

\end{itemize}

\textit{*Courses currently pursuing}
\end{longversion}


\begin{longversion}
\section{Technical Skills}\begin{itemize}
\item \textbf{Programming Languages:}  C, \CPP, Python, Java, JavaScript, Ocaml, NodeJS, VHDL, C\#, Matlab.
\item \textbf{Frameworks:} Tensorflow, Keras, Django, Web2Py, Bootstrap, JQuery

\item \textbf{Programming Environments:} Jupyter, Android Studio, LaTeX, Visual Studio, Xilinx ISE Design Suite

\end{itemize}

\end{longversion}

\section{Extra Curricular Activities}
%\setlength{\leftmargin}{0pt}


\begin{itemize}
    \setlength\itemsep{0em}
    
    \item Developed a chatbot named CampusBot during Code.Fun.Do to fulfil the basic needs of college students.
    \item Attended the St Stephens Model United Nations recreating World War 2 diplomacy over the course of three days.
    \item Attended the Ground Zero Summit and Workshop, the largest grey hat hacking conference in India.
    \item Member of the Development Club at IITD, working to foster the spirit of making open source technology.
    \item Member of the Microsoft Student Partner program with the goal of organizing hackathons and seminars in our campus. 
     
\end{itemize}



\end{document}
